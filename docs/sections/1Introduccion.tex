\section{Introducción}

Pepper es un robot social desarrollado por SoftBank Robotics, diseñado para la interacción humana y ampliamente utilizado en entornos comerciales y educativos \parencite{softbank2023pepper}. Entre otras características físicas, este modelo cuenta con una base móvil y extremidades superiores con un total de 20 grados de libertad, lo que le permite realizar movimientos naturales que intentan emular la expresividad de un ser humano de la manera más natural posible \parencite{softbank2023pepper}. \\

La Facultad de Ingeniería de la Universidad de los Andes actualmente cuenta con dos ejemplares de robot Pepper, llamados Nova y Ópera, principalmente ligados a la iniciativa estudiantil SinfonIA. Estos robots han sido utilizados en diversos escenarios académicos y en eventos promocionales tanto de la iniciativa como de la universidad. Su uso en estos espacios ha demostrado el potencial de la robótica social, perfilando a la Universidad de los Andes como pionera de este campo en Colombia.\\

El éxito de estas aplicaciones ha sido posible gracias a la calibración cuidadosa de los sensores del robot y a la programación manual de su lógica de comportamiento. No obstante, estas tareas requieren una inversión considerable de tiempo y esfuerzo, ya que deben repetirse para cada situación específica. Automatizar estos procesos no solo optimizaría los tiempos de desarrollo, sino que permitiría enfocar los esfuerzos humanos en diseñar experiencias más complejas e innovadoras para la interacción.\\

Las capacidades del robot Pepper se agrupan principalmente en navegación, percepción, comunicación y manipulación. En particular, esta última se centra en el movimiento de los brazos del robot e implica operarlos como manipuladores seriales cuyo efector final es una mano capaz de señalar, agarrar, transportar y sostener objetos. Si bien el sistema operativo del robot permite controlar los ángulos articulares para alcanzar determinadas posiciones, hacerlo de forma manual exige una sintonización meticulosa y propensa a errores, lo que limita su eficiencia y escalabilidad. \\

Debido a lo anterior, se propone emplear técnicas de Aprendizaje por Refuerzo (RL, por sus siglas en inglés) para automatizar la estimación de los ángulos articulares requeridos para alcanzar una posición deseada. En este documento, se presenta un desarrollo que implementa el entrenamiento de modelos de RL a partir del ajuste de hiperparámetros de algoritmos conocidos aplicados sobre una simulación de Pepper con el fin de que aprenda, de forma autónoma, a posicionar sus brazos en coordenadas específicas.\\

La ventaja del uso de RL frente a otras aproximaciones de aprendizaje automático radica en que no se depende de una colección de datos etiquetados para el entrenamiento. Así pues, se busca que el robot aprenda a ajustar sus ángulos articulares a partir de su propia experiencia, según la recompensa o pensalización que reciba en retroalimentación por explorar diferentes configuraciones de ángulos. De este modo, Pepper puede desarrollar políticas de comportamiento que le permitan alcanzar poses precisas de forma eficiente, flexible y adaptable a distintas condiciones del entorno.\\