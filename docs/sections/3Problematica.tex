\section{Problemática y Justificación}

La automatización de sistemas se ha convertido en una de las ramas más importantes en la ingeniería y la industria moderna. Su propósito principal consiste en asegurar que los procesos de dicho sistema se ejecuten correctamente, toda vez que se busca reducir el tiempo que toma completarlos o la cantidad de recursos, tanto energéticos como monetarios, que se consume para realizarlos. En medio de esta optimización, los procesos deben seguir completándose y ejecutándose adecuadamente, incluso sin necesidad de supervisión e intervención humana.\\

En el campo de la robótica, la automatización es fundamental para asegurar que un robot sea capaz de completar satisfactoriamente las tareas para las cuales fue construído. Específicamente, un robot social como Pepper debe ser capaz de realizar tareas asociadas mayormente a la interacción con seres humanos. El sistema operativo del robot ofrece diversas opciones para que los desarrolladores intervengan los parámetros necesarios y permitan que el robot mueva sus brazos, hable o navegue hacia alguna ubicación especificada, para concretar su interacción con un ser humano. Así mismo, Pepper ofrece la posibilidad de recuperar los datos que obtiene de sus sensores y cámaras.\\

Actualmente, la automatización de la navegación de los robots es un problema ampliamente estudiado y que cuenta con diversas propuestas de solución enfocadas en hacer más precisos y eficientes los algoritmos de planeación de trayectoria, los cuales reciben un mapa del espacio en el que se encuentra el robot y retornan una ruta que puede seguir para llegar a un punto de destino específico. De igual manera, las tareas de percepción son estudiadas en el campo de la computación visual por medio de algoritmos de aprendizaje profundo encargados de refinar la detección de obstáculos y la decodificación y clasificación de objetos en el campo visual de una imagen, por ejemplo. Por último, las tareas de comunicación han tenido un impulso sustancial debido a la popularización de los Modelos Extensos de Lenguaje (LLMs, por sus siglas en inglés) y de igual manera, ya son estudiados y ampliamente tratados mediante técnicas de procesamiento de lenguaje natural.\\

La automatización de las tareas de manipulación o, en general, de acomodación de posiciones forma parte del problema de la cinemática inversa. Este tiene soluciones directas y ampliamente conocidas para el caso de robots simples como un robot diferencial, pero se vuelve más difícil en otro tipo de robots con geometrías más complejas. En el caso de un robot Pepper, el problema se complifica debido a que los brazos de un robot Pepper tienen cinco grados de libertad (técnicamente seis si se considera la apertura y cierre de la mano), cada uno. Al contar con un conjunto de eslabones rígidos conectados mediante articulaciones con rangos de operación limitados, un brazo del robot Pepper se puede modelar como un manipulador. \\

El problema de la cinemática inversa es complejo en un manipulador serial de dos eslabones debido a que existen múltiples soluciones para una misma posición. A medida que se incrementan los grados de libertad del manipulador, el problema aumenta de complejidad. Adicionalmente, en el caso de Pepper, otra dificultad adicional surge debido a que algunos de los eslabones del brazo de Pepper tienen dos grados de libertad asociados en lugar de solo uno. \\

A diferencia de las tareas previamente mencionadas, donde existen soluciones estandarizadas y librerías robustas, las aproximaciones tradicionales al problema de cinemática inversa se basan principalmente en métodos numéricos. Estos métodos iterativos intentan estimar ángulos articulares que lleven al efector final a la posición deseada, pero son sensibles a condiciones iniciales, pueden no converger o hacerlo a soluciones no óptimas. Igualmente, presentan limitaciones en escenarios con restricciones complejas o geometría redundante.\\

A raíz de estas limitaciones que presentan los enfoques tradicionales, el uso de técnicas de aprendizaje por refuerzo ofrece una alternativa prometedora. Este enfoque permite al robot aprender, mediante interacción continua con un entorno simulado, a ajustar los ángulos de sus articulaciones para alcanzar posiciones objetivo de manera autónoma y adaptativa. Por otro lado, al no requerir del modelado complejo de la cinemática inversa sino basarse en el aprendizaje de un comportamiento, se espera que arroje soluciones suficientemente robustas, generalizables y eficientes.\\

De esta manera, el presente proyecto surge una propuesta para encontrar una metodología algorítmica y ajustable para abordar la cinemática inversa de un manipulador complejo, como los brazos del Pepper. El planteamiento de esta aproximación contribuye, entonces, al avance de los intentos de automatización de las tareas que puede abordar un robot Pepper. Complementado con otras propuestas existentes de automatizar tareas de navegación, percepción y comunicación, busca reducir el tiempo requerido de trabajo sobre el robot para prepararlo para eventos o para apoyo en escenarios académicos, mejorando la eficiencia del proceso de desarrollo de los equipos de trabajo.\\

